\documentclass{article}
\usepackage{polski}
\usepackage[utf8]{inputenc}
\usepackage{amsmath}
\usepackage{listings}


\title{Dokumentacja projektu}
\date{\today}

\author{
  Michał Padzik\\
  \texttt{padzikm@student.mini.pw.edu.pl}
  \and
  Albert Wolant\\
  \texttt{wolanta@student.mini.pw.edu.pl}
}

\begin{document}

\maketitle
\pagenumbering{gobble}
\newpage
\pagenumbering{arabic}

\section{Opis problemu}

\paragraph{Golf}
Na polu golfowym znajduje się $n$ piłeczek oraz $n$ dołków na piłeczki. Golfiści chcą jednocześnie, każdy swoją piłeczkę, umieścić, w którymś z dołków. W tym celu ustalają między sobą, który celuje do którego dołka, ale w taki sposób, by tory ich piłeczek się nie przecinały, co gwarantuje brak zderzeń piłeczek. Załóżmy, że piłeczki i dołki są punktami na płaszczyźnie oraz, że żadne trzy z tych punktów nie są współliniowe i że tory piłeczek są odcinkami prostej. Przedstaw działający w czasie $O(n^2 log( n))$ algorytm przydzielania piłeczek do dołków tak, aby żadne dwa tory piłeczek nie przecinały się.

\section{Format danych}
W poniższej sekcji znajduje się proponowany format danych wejściowych i wyjściowych razem z przykładami.

\subsection{Dane wejściowe}
Dane wejściowe mają format pliku tekstowego. Struktura tego pliku jest następująca:
\begin{enumerate}
\item Liczba par (golfista, dołek) w nowej linii.
\item Współrzędne $x$ i $y$ golfisty, oddzielone spacją, w nowej linii. Ta linia występuje tyle razy, ile jest golfistów.
\item Pusta liniia.
\item Współrzędne $x$ i $y$ dołka, oddzielone spacją, w nowej linii. Ta linia występuje tyle razy, ile jest golfistów.
\end{enumerate}

\subsection{Dane wyjściowe}
Dane wyjściowe mają format pliku tekstowego. Plik ten składa się z sekwencji następującej postaci:
\begin{enumerate}
\item Współrzędne $x$ i $y$ golfisty, oddzielone spacją, w nowej linii.
\item Współrzędne $x$ i $y$ dołka, do którego strzela powyższy golfista, oddzielone spacją, w nowej linii.
\item Pusta linia.
\end{enumerate}
W pliku występuje tyle sekwencji przedstawionej wyżej postaci, ile jest par (golfista, dołek). Sekwencje są umieszczone jedna po drugiej.

\subsection{Przykład}

\subsubsection{Przykładowy plik wejściowy}

\lstset{
numbers=left,
frame=single
}
\begin{lstlisting}
3
0 0
2.5 3.5
3.5 4.5

1 0
10.45 20.5
4 2
\end{lstlisting}
Plik zawiera dane 3 golfistów o współrzędnych $(0; 0)$, $(2.5; 3.5)$ i $(3.5; 4.5)$ oraz 3 dołków o współrzędnych $(1; 0)$, $(10.45; 20.5)$ i $(4; 2)$.

\subsubsection{Przykładowy plik wyjściowy}
\begin{lstlisting}
0 0
1 0

2.5 3.5
4 2

3.5 4.5
10.45 20.5

\end{lstlisting}
Plik zawiera 3 dopasowane pary. Golfista o współrzędnych $(0; 0)$ strzela do dołka o współrzędnych $(1; 0)$, golfista o współrzędnych $(2.5; 3.5)$ strzela do dołka o współrzędnych $(4; 2)$, a golfista o współrzędnych $(3.5; 4.5)$ strzela do dołka o współrzędnych $(10.45; 20.5)$.

\newpage
\section{Opis rozwiązania}

Poniżej zamieszczamy proponowany algorytm rozwiązania problemu w postaci opisowego pseudokodu:

\paragraph{Dane}
\begin{itemize}
\item Lista $G$ punktów oznaczających pozycje golfistów.
\item Lista $D$ punktów oznaczających pozycje dołków.
\end{itemize}

\paragraph{Algorytm}
\begin{enumerate}
\item $s :=$ punkt o najmniejszej współrzędnej $y$ w listach $G$ i $D$ (jeżeli jest kilka takich punktów to weź ten z najmnjiejszą współrzędną $x$).
\item Złącz pozostałe punkty w listę $C$.
\item Posortuj listę $C$ kątowo, przeciwnie do ruchu wskazówek zegara, względem punktu $s$.
\item Idąc od początku listy $C$ stwórz sumę $S$.
	\begin{enumerate}
	\item $S := 0$.
	\item $c :=$ pierwszy punkt listy $C$.
	\item Dopóki $S\neq-1$
		\begin{enumerate}
		\item Jeżeli $s$ jest dołkiem i $c$ jest golfistą to $S--$
		\item Jeżeli $s$ jest dołkiem i $c$ jest dołkiem to $S++$
		\item Jeżeli $s$ jest golfistą i $c$ jest golfistą to $S++$
		\item Jeżeli $s$ jest golfistą i $c$ jest dołkiem to $S--$
		\item $c:=$ kolejny element na licie $C$.
		\end{enumerate}
	\end{enumerate}
\item Połącz $s$ z punktem poprzedzającym $c$ na liście $C$. Niech ten punkt to $p$.
\item Rekurencyjnie wywołaj algorytm dla punktów poprzedzających $p$ na liście $C$ oraz dla punktów następujących po $p$ na liście $C$.
\end{enumerate}



\section{Analiza poprawności}
Poniżej zamieszczamy dowód, że dopasowanie punktów zdefiniowane w zadaniu zawsze istnieje oraz przedstawiony powyżej algorytm znajduje takie dopasowanie. Będzie to dowód indukcyjny ze względu na liczbę par (golfista, dołek). 
Oznaczmy liczbę tych par w zadaniu przez $n$. 

\begin{enumerate}
\item Jeśli $n=1$, dopasowanie zawsze istnieje w sposób oczywisty poprzez połączenie jedynych dwóch punktów zadania. Algorytm znajdzie takie dopasowanie, ponieważ wybierze jeden z punktów jako $s$ i dołączy do niego drugi jako jedyny punkt listy $C$.
\item Dla $n>1$ zakładamy, że dopasowanie istnieje dla każdego $k<n$ par i algorytm je znajduje.\\
Wykonajmy teraz przedstawiony algorytm bez zejść rekurenycjnych z punktu 6. Po pierwsze, suma $S$ zawsze osiągnie wartość $-1$. Wynika to z faktu, że początkowo w zbiorze jest tyle samo golfistów i dołków. Po wyznaczeniu i wyjęciu ze zbioru punktu $s$, punktów jednej z tych grup jest o jeden więcej niż drugiej. Dodatkowo wśród pozostałych punktów liczniejsze będą punkty innej kategorii niż punkt $s$ (np. jeśli $s$ jest dołkiem to będzie więcej golfistów). W takiej sytuacji, mając na uwadze zasady konstruowania sumy $S$ widać, że zawsze osiągnie ona wartość $-1$, być może po przejżeniu wszystkich punktów. Zatem zawsze wyznaczymy punkt $p$ i otrzymamy $2$ zbiory punktów oddzielone prostą $sp$. W każdym z tych zbiorów jest tyle samo dołków i golfistów i jest tam mniej niż $n$ par.  Zwróćmy uwagę, że wartość $S$ nigdy nie jest mniejsza od $-1$, ponieważ w chwili osiągnięcia $-1$ przeglądanie punktów jest przerywane. Wynika stąd, że przed analizą punktu $p$ suma $S$ była równa $0$. Znaczy to, że wśród odwiedzonych do tej pory punktów jest tyle samo dołków co golfistów. Dodatkowo, ponieważ początkowo w zbiorze grupy golfistów i dołków były równoliczne i punkty $s$ i $p$ należą do różnych grup wiemy, że wśród punktów jeszcze nie odwiedzonych tyle samo jest dołków i golfistów. Oczywiste jest też, że w każdym z tych podzbiorów jest mniej niż $n$ par. Zatem dla każdego z tych zbiorów istnieje odpowiednie dopasowanie i przytoczony algorytm je znajduje. Wynika to z założenia indukcyjnego. Dodatkowo rozwiązania można połączyć bez straty poprawnoci, bo leżą na różnych półpłaszczyznach. Po dodaniu do tego rozwiązania odcinka $sp$ mamy rozwiązanie problemu rozmiaru $n$ i znalezione zostało przedstawionym algorytmem.
\item Z zasady indukcji matematycznej wynika, że przedstawiony algorytm jest poprawny.
\end{enumerate}


\section{Analiza złożonoci}
Niech $n$ oznacza liczbę par (golfista, dołek) w zadaniu.
\begin{itemize}
\item Punkt 1. algorytmu można wykonać w czasie $O(n)$.
\item Punkt 2. algorytmu można wykonać w czasie $O(1)$ przy odpowiedniej implementacji.
\item Sortowanie kątowe w punkcie 3. można wykonać w czasie $O(nlog(n))$.
\item W pesymistycznym wypadku w punkcie 4. trzeba będzie przejżeć całą listę. Trwa to $O(n)$.
\item Dodanie nowej pary do rozwiązania można wykonać w czasie stałym.
\item W pesymistycznym przypadku, czyli wtedy gdy punkt $p$ jest ostatnim na liście $C$, zawsze będzie wykonywane tylko jedno wywołanie rekurencyjne. W takim przypadku podzbiór, dla którego przeprowadzimy to zejście rekurenycjne, ma jedną parę mniej. Zatem pesymistycznie wykonamy $O(n)$ zejść rekurencyjnych.
\end{itemize}
Podsumowując każde wywołanie rekurencyjne ma złożoność  $O(nlog(n))$. W przypadku pesymistym wykonanych będzie $O(n)$ takich wywołań. Zatem pesymistyczna złożoność czasowa całego algorytmu wynosi $O(n^2log(n))$.

\end{document}